% %\documentclass[oneside,final,14pt]{article}
\documentclass[14pt,oneside,final]{extreport}
\usepackage[utf8]{inputenc} 
\usepackage[english,russian]{babel}
\usepackage{vmargin}
\setpapersize{A4}
\setmarginsrb{20mm}{20mm}{20mm}{20mm}{0pt}{0mm}{0pt}{13mm}
\usepackage{indentfirst}
\usepackage{pscyr} % Нормальные шрифты
\DeclareUnicodeCharacter{00A0}{~}
\usepackage[T2A]{fontenc} % Поддержка русских букв
\usepackage{url}
\usepackage{titlesec}

   
\titleformat{\chapter}[display]
{\filright\large}
{} % %{\MakeUppercase{\chaptertitlename} \thechapter}
{8pt}
{\bfseries}{}
     
\titleformat{\section}
{\normalsize\bfseries}
{\thesection}
{1em}{}
     
\titleformat{\subsection}
{\normalsize\bfseries}
{\thesubsection}
{1em}{}
     
% Настройка вертикальных и горизонтальных отступов
\titlespacing*{\chapter}{0pt}{-30pt}{8pt}
\titlespacing*{\section}{\parindent}{*4}{*4}
\titlespacing*{\subsection}{\parindent}{*4}{*4}
    
% всавка изображений
\ifx\pdfoutput\undefined
\usepackage{graphicx}
\else
\usepackage[pdftex]{graphicx}
\fi
% сквозная нумерация
\usepackage{chngcntr}
\counterwithout{figure}{chapter}

\renewcommand{\thefigure}{\arabic{figure}}
\usepackage[tableposition=top]{caption}
\usepackage{subcaption}
\DeclareCaptionLabelFormat{gostfigure}{Рисунок #2}
\DeclareCaptionLabelFormat{gosttable}{Таблица #2}
\DeclareCaptionLabelSeparator{gost}{~---~}
\captionsetup{labelsep=gost}
\captionsetup[figure]{labelformat=gostfigure}
\captionsetup[table]{labelformat=gosttable}
\renewcommand{\thesubfigure}{\asbuk{subfigure}}

\linespread{1.3}
\usepackage[toc,page]{appendix}
    \renewcommand\appendixname{Приложение}
    \makeatletter
    \def\redeflsection{\def\l@section{\@dottedtocline{1}{1.5em}{7.8em}}}
    \renewcommand\appendix{\par
    \setcounter{section}{0}%
    \setcounter{subsection}{0}%
    \def\@chapapp{\appendixname}%
    \addtocontents{toc}{\protect\redeflsection}
    \def\thesection{\appendixname\hspace{0.2cm}\@arabic\c@section}}
    \makeatother


\sloppy


\begin{document}
	\renewcommand*\contentsname{\hfill Содержание \hfill}
	\tableofcontents
	
	\chapter{Введение}
	Эксплуатационные характеристики изделий машиностроения в большой мере зависят от качества изготовления деталей, составляющих изделие. Качество изготовления определяется степенью соответствия параметров готовой детали и параметров, достижение которых требуется документацией. В роли такого параметра может выступать точность изготовления геометрических размеров, определяемая допуском. 
	Погрешности различных видов препятствуют абсолютно точному воспроизведению заданных параметров. Например, на точность изготовления деталей влияют погрешности базирования, измерения, настройки инструмента и т.д. При обработке нежестких деталей на первый план выходят погрешности, связанные с деформацией самой детали.
	Рассмотрим, например, обработку тонкостенной детали в трехкулачковом патроне токарного станка. После закрепления такой детали силы, действующие со стороны кулачков,  деформируют заготовку таким образом, что профиль заготовки отличается от идеально цилиндрического. В процессе резания  инструмент неравномерно снимает припуск по окружности заготовки. Дополнительно заготовка деформируется под действием силы резания, а также в связи с неравномерностью нагрева. После окончания обработки и снятия усилий закрепления отклонение от круглости заготовки достигает величин, сравнимых с величиной допуска и в зависимости от конкретных условий может превышать его.
	На практике в случае обработки нежестких деталей применяется специальное технологическое оборудование, позволяющее уменьшить возникающие деформации. Применение сырых кулачков, растрачиваемых под диаметр заготовки - один из способов уменьшения деформаций закрепления. После растачивания кулачок охватывает деталь по большей площади. Также используется обработка с технологическим заполнителем (например, легкоплавким материалом), который увеличивает жесткость детали на время обработки и выплавляется после окончания обработки. Однако, оба способа ведут к увеличению затрат на производство. Первый подразумевает наличие определенного набора кулачков, расточенных под различные диаметры и, таким образом, дополнительно снижает гибкость производства, ограничивая его имеющейся номенклатурой оснастки. Второй способ ведет к расходу дополнительного материала. 
	Ввиду указанных ограничений на существующие способы обработки предлагается рассмотреть еще один метод - подбор режимов обработки и условий закрепления таким образом, чтобы обеспечить изготовление размеров в рамках допуска. Предполагается, что такой метод может найти применение прежде всего в единичном и мелкосерийном производстве, располагающим как правило только универсальным оборудованием. Суть подхода заключается в том, чтобы заранее, на этапе разработки технологической документации определить режимы резания, менее эффективные с точки зрения производительности, но оптимальные с точки зрения точности изготовления. С уменьшением сил резания уменьшаются и деформации заготовки а также требуемые усилия закрепления, но с другой стороны увеличивается время обработки.  Подобрав баланс между скоростью и точностью обработки технолог имеет возможность назначить режимы резания, приемлемые для изготовления заданной детали без привлечения дополнительной технологической оснастки. 
	    Для поддержки решения о назначении режимов резания предлагается разработать информационную систему, позволяющую анализировать деформации заготовки при заданных геометрических параметрах и режимах резания. Такой инструмент можно использовать для последовательной проверки ряда значений параметров процесса и выбора наиболее рациональных. 
	Система, спроектиованная в соответствии с современными тенденциями в разработке ПО станет эффективным инструментом для назначения режимов обработки. Объектно-ориентированная модульная архитектура такой системы позволит расширять её функционал в случае возникновения новых требований к применению системы, а независимость от конкретной операционной системы позволит легко внедрить такие системы в процесс разработки технологической документации на предприятиях безотносительно конкретной существующей информационной среды.
	
	
	\chapter{Техническое задание}	
	
	\section{Основания для разработки}
	Основанием для разработки системы являются документы:
	\begin{itemize}
	\item Задание на выполнение дипломного проекта
	\item Календарный план на выполнение дипломного проекта
	\end{itemize}
	
	\section{Назначение разработки}
	    Функциональным назначением разработки является обеспечение интегрированной рабочей среды для гибкого моделирования процессов деформирования тонкостенных заготовок в процессе их токарной обработки. Под гибкостью понимается возможность изменения параметров процесса в зависимости от расчетного случая и конфигурации заготовки. Эксплуатационным назначением является обеспечение инструментального средства для последовательного определения рациональных режимов резания тонкостенных заготовок и автоматизация сопутствующих расчетов.
	
	\section{Требования к функциональным характеристикам}
	    Система информационной поддержки должна обеспечивать выполнение следующих функций:
	Расчет деформаций тонкостенной заготовки при токарной обработке с закреплением в кулачковом патроне.
	Автоматизация необходимых сопутствующих расчетов (режимов резания)
	Изменение параметров рассматриваемого процесса: геометрии заготовки  и оснастки и значений силовых факторов (силы резания и закрепления)
	Представление результатов расчета в виде графиков или таблиц деформаций
	Генерация отчета по результатам расчета
	Вывод дополнительных информационных отладочных сообщений 
	\section{Требования к надежности}
	    Разрабатываемая система должна обеспечивать надежную работу в условиях ошибочного пользовательского ввода.
		\section{Условия эксплуатации}
	    Температура окр. среды, влажность, давление и т.д
		\section{Требования к составу и параметрам технических средств}
	    ОЗУ, ПЗУ, монитор и т.д.
		\section{Требования к информационной и программной совместимости}
	Windows, Linux, и т.д.
	\section{Требования к маркировке и упаковке}
	     Не предъявляются
	\section{Требования к транспортированию и хранению}
	    Не предъявляются
	\section{Требования к программной документации}
	    Не предъявляются
		\section{Технико-экономические показатели}
	    ? ? ?
		\section{Стадии и этапы разработки}
	    ? ? ?
	\section{Порядок контроля и приемки}
	    ? ? ?
		\section{Приложения}
	    ? ? ?
		\section{Предпроектное исследование}
	    ? ? ?
	\chapter{Разработка концепции автоматизированной системы}
	\section{Принцип моделирования}
	Согласно техническому заданию система должна обеспечить расчет деформаций тонкостенной заготовки в процессе токарной обработки. Токарная обработка характеризуется большим разнообразием возможных вариантов осуществления в зависимости от обрабатываемой поверхности и способа закрепления. Разработать и запрограммировать адекватную модель, основанную исключительно на математическом представлении известных теоретических закономерностях сопротивления материалов для каждого из возможных вариантов - чрезвычайно сложная задача с технической точки зрения. 
	Известно, что современные CAE системы позволяют успешно решать задачи определения деформаций и напряжений для объектов практически неограниченной сложности благодаря использованию метода конечных элементов. После разбиения рассматриваемых объектов на конечные элементы определенного типа и наложении ограничений на эти элементы согласно закономерностям предметной области задача сводится к решению системы из большого количества уравнений. Успех решения полученной системы уравнений как правило определяется только отведенным на решение временем работы ЭВМ. 
	Тем не менее, сама по себе реализация метода конечных элементов в разрабатываемом продукте не возможна из-за большой трудоемкости и требует огромной исследовательской и проектной работы целого коллектива профессионалов. Таким образом, наилучшим выходом из сложившейся ситуации представляется использование готовой CAE системы. Использование системы МКЭ расчета возможно, если она отвечает следующим трем основным требованиям:
	Предоставляет возможность расчета заранее составленной модели под управлением внешнего процесса операционной системы (в роли которого будет выступать разрабатываемая система анализа деформаций)
	Имеет механизмы параметризации модели. В наилучшем случае параметризация должна достигаться благодаря выполнению пользовательских сценариев.
	Позволяет получить результат расчета в виде графиков, диаграмм напряжений и деформаций а также в числовом виде для каждого элемента модели.
	В рамках перечисленных соображений можно сформулировать концепцию системы следующим образом. На первом этапе система собирает информацию о значениях параметров модели через графический пользовательский интерфейс. По завершению ввода функция системы состоит в формировании задания для используемой готовой CAE системы. На этапе подготовки задания производится дополнительны расчет необходимых величин, если они не были непосредственно указаны на первом этапе и могут быть получены расчетным путем. Третий этап состоит в вызове расчетного ядра CAE системы по сформированному заданию. По завершению обработки модели разрабатываемая система агрегирует результаты и отображает их пользователю, при необходимости генерируя отчет.
	Таким образом, основная идея состоит в расчете модели обработки с использованием сторонней CAE системы. Предполагается, что роль разработчика моделей будет выполнять эксперт, как правило не являющийся пользователем системы. В задачи эксперта входит описание нового расчетного случая при помощи инструментов, предоставляемых конкретной выбранной системой МКЭ расчета и другие необходимые работы по подготовке системы. Готовая модель предоставляется в распоряжению пользователю, в задачи которого входит выбор наиболее подходящей модели из множества имеющихся и её загрузка в разрабатываемую систему поддержки.
	\section{Принцип модульности}
	Уже на этапе формулирования концепции системы становиться видна её большая сложность. Для того чтобы сохранить контроль над расширяющейся по мере разработке системы и не допустить её деградации следует определиться с принимаемыми для этого мерами и также включить их в концепцию системы, т.к. принимаемые меры коренным образом повлияют на процесс проектирования. 
	В соответствии с накопленным мировым опытом в разработке программного обеспечения одним из наиболее жизнеспособных способов контроля сложности является разработка приложения в соответствии с принципами  объектно-ориентированного программирования. Во-первых,  ООП методология позволяет добиться большого процента повторного использования кода благодаря принципам наследования и полиморфизма. Во-вторых, принципы абстракции и инкапсуляции облегчают задачу программиста т.к. существенно ограничивают область кода, который влияет на рассматриваемый участок программы и позволяют формулировать мысли в терминах предметной области. 
	Разрабатываемая система сложна не только по причине большой трудоемкости процесса разработки, но также и процесса поддержки готовой системы. В случае возникновения новых функциональных требований, не покрытых техническим заданием на первом этапе, потребуется совершить большой объем работ, если заранее не предусмотреть модульность системы. Модульность системы позволит “присоединить” к готовой системе недостающие функциональные элементы - модули, без переработки какой-либо существенной части самого приложения. 
	Как показано выше, сам по себе переход от структурного программирования к объектно-ориентированному - большой шаг вперед. Однако, такой шаг ещё не гарантирует структурированной модульной архитектуры приложения. Отдельный класс хоть и инкапсулирует некоторые данные, тем не менее не является модулем в смысле всего приложения. Для достижения этой цели потребуется применение специальных мер, зависящих от выбранного языка программирования. Такие меры могут представлять, например, использование определенного набора шаблонов проектирования, в случае реализации модульности системы своими силами, или использования одного из многочисленных фреймворков в другом случае.  
	
	
	\section{Принцип комплексности}
	    Принцип модульности оставляет широкие возможности для расширения функционала системы даже после выпуска готовой системы. Однако такой возможностью следует пользоваться в пределах разумного. Не следует исходить из этой концепции для оправдания отсутствия каких-либо востребованных функций. В контексте текущего проекта должен быть разработан исходный набор модулей, обеспечивающий выполнение большинства функций, которые могут понадобиться пользователю при работе в рамках рассматриваемой предметной области. Это означает что следует комплексно подойти к решению проблемы. 
	В рамках базового набора модулей согласно техническому заданию должны обязательно присутствовать модуль для расчета режимов резания, модуль графического отображения результатов и модуль генерации отчетов. Также следует включить в основной состав модули просмотра трехмерного изображения рассчитываемой модели и вывода текстовой информации, формируемой CAE системой. 
	Кроме самого факта наличия модулей в системе необходимо обеспечить взаимный обмен информацией между ними. Например, передачу значений, полученных в модуле расчета режимов резания в модуль параметризации модели. Этот прием позволит избавить пользователя от случайных ошибок при копировании и сделает работу с системой более удобной. Согласно идее комплексности, полезной особенностью может стать предоставление справочных данных для выбора в качестве значений вводимых параметров. 
	Таким образом, разрабатываемая система должна стать результатом многосторонней проработки проблемы. 

	\section{Принцип независимости}
	При разработке системы уже на самых ранних стадиях необходимо ориентироваться на возможное многообразие используемых на  предприятиях программных и аппаратных средств. Известно, что для  современного рынка информационных технологий характерна большая вариативность используемого программного обеспечения. В рамках проекта интерес представляют операционные системы и системы инженерных расчетов (CAE). Для разных предприятий сочетания конкретных решений в этих областях могут быть различными, например, роль операционной системы может выполнять какая либо версия Microsoft Windows или  Linux. В роли CAE системы может использоваться Ansys, Abaqus, в некоторых  случаях Siemens NX или SolidWorks. 
	В идеальном случае следует ориентироваться на независимость проектируемого приложения от индивидуальных особенностей какой-либо внешней системы. Это позволит расширить границы применения программы и облегчить задачу её освоения в рамках существующей информационной среды. Удовлетворить условиям концепции независимости поможет правильный выбор средств разработки а также определенные проектные меры, например, разработка межсистемного взаимодействия на основе соглашений (интерфейсов). 
	
	
	
%	\chapter{Предпроектное исследование}
%	
%		\section{Введение}
%		Д
%
%		 \begin{figure}[!h]
%		 \begin{center}
%		 \includegraphics[scale=0.07]{img/ProjectHistory} 
%		 \end{center}
%		 \caption{История развития работ}
%		 \label{ris:ProjectHistory}
%		 \end{figure}
%		
%	
%	\section{Выбор программных средств}
%		\subsection{Выбор системы моделирования}
%		
%	\renewcommand{\bibname}{\centerline{\large{Список литературы}}}
%	\bibliographystyle{utf8gost705u} 
%	\bibliography{biblio}
%	\addcontentsline{toc}{chapter}{Список литературы} % добавляем пункт "Литература" в
%	\nocite{Exp1}
%	\nocite{OOP}
%	\nocite{Java1}
%	\nocite{Java2}
%	\nocite{Exp2}
	
\end{document}